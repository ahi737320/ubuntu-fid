\documentclass[letterpaper,10pt,titlepage]{article}

%url support
\usepackage{url}
\usepackage[pdftex]{hyperref}
\usepackage{nameref}
\hypersetup{
    colorlinks,%
    citecolor=black,%
    filecolor=black,%
    linkcolor=black,%
    urlcolor=black
}

%underscore w/o escape
\usepackage{underscore}

%pdf metadata and sizing
\usepackage{ifpdf}
\ifpdf
\pdfinfo
{ /Title (NVIDIA Ubuntu packaging guide)
  /Author (Mario Limonciello)
  /CreationDate (D:20090824042607) % this is the format used by pdf for date/time
}
\fi

\title{\textbf{NVIDIA Ubuntu packaging guide}}
\author{Mario Limonciello\\ Mario\_Limonciello@Dell.com}
\date{\today}

\begin{document}
\maketitle

\tableofcontents
\newpage

\section{Introduction}
In the process of doing testing and development on Dell machines that contain NVIDIA graphics, you may need to update the NVIDIA graphics driver
to one that is newer than already released.  This guide will help you through that process.

\section{Check Versions}
You'll want to be operating from an Ubuntu install to do these things.
\begin{enumerate}
\item Visit \url{https://launchpad.net/ubuntu/+source/nvidia-graphics-drivers-180}. This URL may change if the "source" package name changes at some point.
\item Click on the latest version number.
\item Is this version greater than or equal to the version you need?
  \begin{itemize}
    \item \textbf{Yes}, goto \nameref{grabbing}
    \item \textbf{No},  goto \nameref{updating}
  \end{itemize}
\end{enumerate}

\section{Grabbing a Build} \label{grabbing}
Start out by clicking the i386 link of the builds on the right.  From that \textit{built files} section, you will need these packages:
\begin{itemize}
\item nvidia-185-kernel-source
\item nvidia-185-libvdpau
\item nvidia-185-modaliases
\item nvidia-glx-185
\end{itemize}
Install these files on your system and test.  If your tests are successful, goto \nameref{building}.

\section{Updating the version} \label{updating}
\begin{enumerate}
\item From the weblink you started at, download the \textbf{.orig.tar.gz}, \textbf{.diff.gz} and \textbf{.dsc} files.
\item Install the dpkg-dev and devscripts packages without recommends.
  \newline   
  \texttt{ apt-get install dpkg-dev devscripts --no-install-recommends}
\item Extract the ubuntu package using this command:
  \newline 
  \texttt{ dpkg-source -x *.dsc}
\item Switch into the newly created directory
\item Delete any existing .run files
\item Download the NVIDIA 32 and 64 bit drivers from the NVIDIA website into the root of the newly created directory.
\item Edit debian/upstream_info.  Particularly the \textbf{RELEASE}, \textbf{NEXTVER}, and any references to \textbf{pk0/pkg1} or \textbf{pkg1/pk2}.
\item Update the debian/changelog using the dch tool: 
  \newline 
  \texttt{dch -v VERSION-0ubuntu0}
  \begin {itemize}
    \item \textit{VERSION} is the upstream version.
    \item \textit{0ubuntu0} is explicitly what it said.  So for 185.18.36 it would be:
    \item \texttt{dch -v 185.18.36-0ubuntu1}
    \item  When the editor opens up, note that this is a \textit{New Upstream Version}.
    \item  Make sure the release at the top (jaunty, karmic, hardy etc) is correct.
    \item Save the file.
  \end{itemize}

\item Install the dependencies necessary to build:
  \newline
  \texttt{ apt-get build-dep nvidia-graphics-drivers-180}
\item Build the package
  \newline
  \texttt{ debuild}
\item From the generated debs you will need these:
  \begin{itemize}
    \item nvidia-185-kernel-source
    \item nvidia-185-libvdpau
    \item nvidia-185-modaliases
    \item nvidia-glx-185
  \end{itemize}
\item Test the packages.
\item If successful, goto \nameref{building}
\end{enumerate}


\section{Building an iDrive Release} \label{building}
\begin{enumerate}
\item Create this folder structure for the release
 \texttt{mkdir -p debs/nvidia}
\item Copy all of the necessary debs into that folder
\item Create a gzipped tarball

\end{enumerate}
\end{document}
