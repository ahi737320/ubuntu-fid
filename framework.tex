\documentclass[letterpaper,10pt,titlepage]{article}

%url support
\usepackage{url}
\usepackage[pdftex]{hyperref}
\hypersetup{
    colorlinks,%
    citecolor=black,%
    filecolor=black,%
    linkcolor=black,%
    urlcolor=black
}

%underscore w/o escape
\usepackage{underscore}

%pdf metadata and sizing
\usepackage{ifpdf}
\ifpdf
\pdfinfo
{ /Title (Ubuntu OEM Recovery framework)
  /Author (Mario Limonciello)
  /CreationDate (D:20080530042607) % this is the format used by pdf for date/time
}
\fi

\title{\textbf{Ubuntu OEM Recovery framework}}
\author{Mario Limonciello\\ Mario\_Limonciello@Dell.com}
\date{\today}

\begin{document}
\maketitle

\tableofcontents
\newpage

\section{Introduction}
A common issue for OEMs who want to ship Ubuntu on their machines is developing a sustainable method for creating recovery disks for their customers.  The problem is amplified when OEMs attempt to make these recovery disks emulate the process used for creating the factory installation.  Multiply this by the large number of variable hardware and software configurations offered at system creation time and the problem becomes unmaintainable.
\\
The logical solution is to create a framework that can be reused and adapted to work with a variety of teams, ensuring that a single team's work will not interfer with or detrimentally affect another team's workflow.

\section{Framework overview}
This framework provides a solution that allows OEMs to provide hooks into each portion of the installation.  These hooks can be workarounds, drivers for hardware enablement, software, documentation, and localization.  The same hooks are then replicated on the resultant system allowing users to create a recovery disk with all of the same benefits.

\subsection{Basic layout}
The framework itself is laid out in the following structure:
\begin{itemize}
\item ./grub
\item ./docs
\item ./preseed
\item ./scripts
\item ./scripts/chroot-scripts
\item ./scripts/chroot-scripts/os-pre
\item ./scripts/chroot-scripts/fish
\item ./scripts/chroot-scripts/os-post
\item ./debs
\item ./debs/main
\item ./isolinux
\end{itemize}

This is intended to supplement the existing directory structure of an Ubuntu DVD.  Each of these directories (and the scripts in them) will be explained as they are brought up.

\section{Factory Installation}
When a standard Ubuntu installation is performed, the contents of the DVD are copied onto a system.  A few minor modifications are made due to the hardware and input parameters related to localization during installation.\
\
For this factory installation, the install is similar to a standard Ubuntu installation, however hooks have been added to perform operations before install begins, modify input variables during install, and add on additional changes to the resultant system.

\subsection{Preparing the installation}
To get started, checkout a copy of the framework and download a copy of the latest Ubuntu DVD.  You will need a blank hard drive to work with as well.
\begin{enumerate}{}
\item Set up the partition structure on the hard drive.  It is expected that you have 2 small partitions with the rest of the drive unallocated.  The first partition is intended for diagnostic utilities and should be 50-100 mb.  The second partition is your recovery partition.  You should set it to be 5 GB and either fat32 or ext3.
\item Download an Ubuntu DVD and extract it's contents to the recovery partition.  Be sure to extract the hidden .disk directory too   This is critical.
\item Download a copy of the OEM framework and extract it "on top" of the extract recovery partition.  There are a few files that need to be overwritten.
\item Prepare the drive to boot onto this second partition.  This can be achieved using a DOS filesystem and linld, a linux filesystem and kexec, or by installing GRUB to the drive.  Do whatever is most conducive to your environment.  Use this kernel command line to get started:
\\
\texttt{kernel /casper/vmlinuz boot=casper edd=on preseed/file=/cdrom/preseed/oem.seed noninteractive noprompt}
\end{enumerate}

\subsection{Booting the installation}
The installation begins by passing a kernel command line to the kernel shipped with the Ubuntu DVD.  This command line will generally look something like this:
\\
\\
\texttt{kernel /casper/vmlinuz boot=casper edd=on preseed/file=/cdrom/preseed/oem.seed noninteractive noprompt}
\\
\\
Let's break up that command line to explain what each portion is doing.
\\
\\
\begin{tabular}{|l|l|}
\hline \textbf{Option} & \textbf{Purpose} \\
\hline kernel & The option to menu.lst to indicate we are loading a kernel. \\
\hline /casper/vmlinuz &  The kernel image we are loading. \\
\hline boot=casper & Indicates that this is a live disk booting and to use unionfs. \\
\hline edd=on & Turns on Enhanced Disk Drive support, very useful in multi disk boxes. \\
\hline preseed.. & The seed file we are passing to the Ubuntu installer. \\
\hline noninteractive & Indicates that we are using the non-GUI installer. \\
\hline noprompt & Tells the system to not prompt at the end of install, but just reboot. \\
\hline
\end{tabular}
\\
\\
\\
The non GUI installer was selected in case the event comes up that video hardware is not supported in X until after postinstall scripts are complete.  If you are sure that video hardware is supported by X in all scenarios, you can replace \textbf{noninteractive} with \textbf{automatic-ubiquity}.
\\
\\
A more complete summary of kernel options available during and related to installation is available at \url{https://wiki.ubuntu.com/DesktopCDOptions}

\subsection{Preboot phase}
As soon as the kernel loads, casper will load the contents of the oem.seed file.  It will use this to determine what the \texttt{early_command} is.  The early command starts immediately after the kernel loads but before all of the init scripts start.  The current early command will call \texttt{scripts/bootstrap.sh}.
\\
This script will then load the OEM framework environment, \texttt{environ.sh} and then individually launch all scripts listed in scripts/bootstrap.  If a command in one of the scripts has a nonzero return value, \texttt{scripts/bootstrap/FAIL-SCRIPT} will be called.  If all of the scripts complete successfully, \texttt{scripts/bootstrap/SUCCESS-SCRIPT} will be called.  Cleanup and recovery options in the fail and success scripts are fully customizable.
\\
\begin{list}{}
\item \texttt{05-CONFIRM-REINSTALL.sh} checks for the existence of the \textbf{REINSTALL} kernel parameter to determine if this is a reinstallation.
\\
\item \texttt{06-CONFIRM-DVD.sh} checks for the existence of the \textbf{DVDBOOT} kernel parameter to determine if this is a DVD being booted.

\item \texttt{10-format.sh} formats the disks to start.  This ensures that the installation will always look the same when the actual Ubuntu install process gets started.  Doing this in all cases (Factory, Reinstall from HD, Reinstall from DVD) will ensure a successful and standard installation.
\end{list}

\subsection{Install phase}
After the \texttt{early_command} completes, the Ubuntu installation will begin.  All of the data that was read in oem.seed will be applied to the normal input parameters for the Ubuntu installer, Ubiquity.  The exact contents of this file will be explained later.
If the installation is successful, the \texttt{success_command} from oem.seed will run.  If the installation failed, the \texttt{failure_command} from oem.seed will launch.

\subsection{Post-install phase}
The postinstallation script is launched as soon as the installation is complete, whether it succeeds or fails.  If the installation failed for any reason, the fail script \texttt{scripts/chroot-scripts/FAIL-SCRIPT} will launch.  In this script you can perform cleanup or another method to analyze what failed in more detail.
\\
\\
In the event of success, this script will prepare the post installation environment similar to how the preinstall script did.  The difference is that all of the scripts launched in postinstall are launched on the resultant system.  This means that you can perform package installation, modify configuration files, and tweak anything as necessary in the environment.  If any script in here fails with a return code greater than 0, \texttt{scripts/chroot-scripts/FAIL-SCRIPT} will launch.
\\
The post-installation is broken into three subphases.

\subsubsection{os-pre}
The \textit{os-pre} phase is the first of the three phases.  It is intended to handle files and functions that would break if interfered with by hardware or software enablement teams at inappropriate times.
\\
\\
Currently two scripts are contained in here:
\begin{list}{}
\item \texttt{10-grub.sh} will add a menu option to the GRUB menu.lst for a hard drive based recovery.  If another team decides that additional options must be added to menu.lst, this ensures that the OS requirement is placed first.
\item \texttt{80-install-debs.sh} will install debian archives that provide dependencies for other steps, as well as archives that don't need to be installed in a particular order.
\\
\end{list}
Again, this section is intended to be used by an OS development team for items related to the OS requirements.  Hardware and software enablement will happen in the other phases.

\subsubsection{fish}
The \textit{fish} phase is the second of the three phases.  It is intended to handle hardware and software enablement.  Teams that need to modify configuration files for default hardware behavior or add drivers should do so here.  Also, teams that need to add extra software that did not fall in the \textit{os-pre} step (possibly due to dependency resolution issues) can do so here as well.  The framework ships with this directory empty, but some examples are available for things that can go in here.

\subsubsection{os-post}
The \textit{os-post} phase is the last of the three phases.  It is intended for final cleanup steps.    The OS team can include scripts here that will be OS wide workarounds as well as final cleanup steps that must happen to prepare the seal on the box.
\\
\\
Shipped here are 4 starting scripts.
\begin{list}{}
\item \texttt{50-handle-docs.sh} is intended to allow documentation and translation teams to include important documents with the base installation.  To prevent having to provide a high bar for them to learn how to package these, this script will simply copy over  relevant documentation onto the installation in a standard location.
\item \texttt{60-fix-apt-sources.sh} is used to add additional vendor specific APT sources.
\item \texttt{90-oem-prepare.sh} is used to put the machine into OEM mode for the first boot.  This causes the machine to ask for a user name and password on the first boot.
\item \texttt{99-active-partition.sh} sets the active bootable partition on the system.  This was intentionally made the last step in case anything failed earlier due to a power outage.  If the step is not run, the system will boot up to the recovery partition and the entire process can start over.
\\
\end{list}

\section{Recovery installation}
During installation, grub is installed to the recovery partition and a default menu.lst is shipped with the OEM framework.   If during boot, a user selects the option to reinstall their system, GRUB will chainload to the second partition.  This menu.lst is presented to the user giving them the option to recover.  The entire process will proceed just like a factory install with the exception of an added boot parameter, \textbf{REINSTALL}.
\\
\\
Earlier we skipped over a script, \texttt{05-CONFIRM-REINSTALL.sh}.  This script checks for the extra boot parameter, and presents the user with a dialog ensuring that they want to wipe the disk.  This script comes before \texttt{10-format.sh}.  If they don't agree to the dialog, the machine will reboot preventing any data loss.

\section{DVD installation}
Within the framework, a directory isolinux/ is shipped that provides an extra menu option to the standard Ubuntu DVD boot menu.
Similar to the HD based recovery, the DVD based recovery provides an extra kernel command line parameter, \textbf{REINSTALL}.
DVD boots also provides the kernel command line parameter \textbf{DVDBOOT} indicating that it is being ran from a DVD.
\\
\\
When \texttt{06-CONFIRM-DVD.sh} is ran, the contents of the MBR and the recovery DVD are copied to the user's system.  The rest of the installation then proceeds like normal, but instead off of the DVD.  This is a little bit slower, but just as functional.
\\
DVD creation will be explained later on.


\section{Use case examples}
In the Examples Directory you will find a handful of real life examples that have been used with this framework.  They have been used for hardware enablement when all the changes could not be easily represented in a debian archive.
\\
If you have a use case that you have found this particularly useful for, please submit it back to this project.  Adding it to the examples section will benefit all parties.
\subsection{Hardware enablement}
There are two exampless of hardware enablements:
\begin{list}{}
\item \textbf{70-nvidia-driver.py} installs an NVIDIA proprietary graphics driver, removes the warning from Jockey, and then activates it in the xorg.conf.  The debs for the NVIDIA graphics driver are not included on the Ubuntu DVD, so they are shipped with this script.
\item \textbf{71-thinkfinger.py} will install and activate thinkfinger support.  The thinkfinger debs come with the Ubuntu DVD so they can be installed from there.  The script also turns on the PAM section to allow swiping to login.
\end{list}

\subsection{Software enablement}
There are also two examples of software enablement:
\begin{list}{}
\item \textbf{50-install-DKMS.sh} sets up the DKMS package.  These debs don't come on the DVD so they are provided independently.  Arguably this fits better in os-pre because some hardware enablement relies on it.
\item \textbf{90-add-lbm.sh} add the linux-backports-modules package to the system.  This package provides updated drivers and firmware for certain cards.
\end{list}

\subsection{Bug workarounds}
\textbf{85-no-ehci.sh} shows how to work around a bug that came with the OS that prevents proper functionality.  In the event that it's too late or complex to fix a bug directly in the OS, this is a good example of how to script a workaround into the system.  It modifies just the necessary files and in a fashion that would not cause harm later when packages are updated.

\section{Debugging}
The most important aspect of debugging is identifying which phase of the installation needs to be debugged.  The methodology used for each phase can vary depending on the problem you are trying to solve.

\subsection{Boot problems}
If you have an underlying kernel problem, you will likely get a panic directly upon boot.  If you don't, you may be dropped down to a console at which point you can look at the output of \texttt{dmesg} for more information.  If you are discovering problems during this phase, it's important to file bugs on these issues early as they are most critical and may require a significant amount of time to fix.

\subsection{Preboot phase}
If the preboot phase is failing, you will be dropped to a console.  Debugging this can be done by adding echo statements to the scripts and looking at the output of \texttt{casper.log}.  Alternatively, you can handle debugging directly in your \texttt{FAIL-SCRIPT}.

\subsection{Install phase}
Debugging the install phase is likely to be the most difficult because problems can either be in the implementation of the preseed file or a bug directly in the installer.  The installer will always place debug output in \texttt{/var/log/syslog} and possibly \texttt{/var/log/installer/debug} depending on the type of crash you encounter.
\\
\\
A useful debug strategy is to boot up a live disk and attempt to simulate an installation environment.  You can perform all of the steps normally done in your preboot phase manually.  After doing this, you can load your preseed file:
\\
\\
\texttt{cat oem.seed | sudo debconf-set-selections}
\\
\\
Once the preseed file is loaded, you can launch Ubiquity in \textbf{debug} mode.
\\
\\
\texttt{ubiquity -d noninteractive}
\\
\\
The logging will be much more verbose in \texttt{/var/log/installer/debug}.  Be sure that if you have any bugs about the installer to file to include your oem.seed in the bug and any of the installation logs that you can provide.

\subsection{Post Install phase}
The post install phase would be most likely to have problems due to unknown interactions of different team's post installation scripts.
Debugging can be performed interactively or retroactively by looking at logs.  The post install phase will log all activity to \texttt{/var/log/installer/chroot.sh.log} on the resultant system.
\\
\\
You can also interactively prevent each script from launching one by one by creating a file \texttt{/tmp/superhalt.flg} on the live filesystem.  This will create a flag, \texttt{/tmp/halt.flg} before each script runs.  When you want the next script to launch, just remove \texttt{/tmp/halt.flg}.  If you would like all of the scripts to continue on, remove both \texttt{/tmp/superhalt.flg} and \texttt{/tmp/halt.flg}

\section{Preseeding}
Each of the items in the preseed file provided with the framework is documented, but if you would like to learn more in depth you can  look at the Ubuntu preseeding guide at \url{https://help.ubuntu.com/8.04/installation-guide/i386/appendix-preseed.html}.

\section{DVD Creation}
For DVD creation, the current framework expects that you will burn a bootable DVD that loads the standard isolinux setup shipped in the framework.
\\
\\
Additionally, you will have to provide a copy of  the system's MBR in a file titled \texttt{mbr.bin} in the root of the DVD image and a gzipped copy of the utility partition in a file titled \texttt{upimg.bin} in the root of the DVD image.  You can choose to remove these portions of the framework as you see fit.
\\
\\
An example application for creating these types of DVDs is available in the Ubuntu archive.  Look at the project entitled \textbf{dell-recovery}.

\section{License}
The framework is covered by the GNU General Public License.
\begin{center}
{\parindent 0in

Version 2, June 1991

Copyright \copyright\ 1989, 1991 Free Software Foundation, Inc.

\bigskip

51 Franklin Street, Fifth Floor, Boston, MA  02110-1301, USA

\bigskip

Everyone is permitted to copy and distribute verbatim copies
of this license document, but changing it is not allowed.
}
\end{center}

\begin{center}
{\bf\large Preamble}
\end{center}


The licenses for most software are designed to take away your freedom to
share and change it.  By contrast, the GNU General Public License is
intended to guarantee your freedom to share and change free software---to
make sure the software is free for all its users.  This General Public
License applies to most of the Free Software Foundation's software and to
any other program whose authors commit to using it.  (Some other Free
Software Foundation software is covered by the GNU Library General Public
License instead.)  You can apply it to your programs, too.

When we speak of free software, we are referring to freedom, not price.
Our General Public Licenses are designed to make sure that you have the
freedom to distribute copies of free software (and charge for this service
if you wish), that you receive source code or can get it if you want it,
that you can change the software or use pieces of it in new free programs;
and that you know you can do these things.

To protect your rights, we need to make restrictions that forbid anyone to
deny you these rights or to ask you to surrender the rights.  These
restrictions translate to certain responsibilities for you if you
distribute copies of the software, or if you modify it.

For example, if you distribute copies of such a program, whether gratis or
for a fee, you must give the recipients all the rights that you have.  You
must make sure that they, too, receive or can get the source code.  And
you must show them these terms so they know their rights.

We protect your rights with two steps: (1) copyright the software, and (2)
offer you this license which gives you legal permission to copy,
distribute and/or modify the software.

Also, for each author's protection and ours, we want to make certain that
everyone understands that there is no warranty for this free software.  If
the software is modified by someone else and passed on, we want its
recipients to know that what they have is not the original, so that any
problems introduced by others will not reflect on the original authors'
reputations.

Finally, any free program is threatened constantly by software patents.
We wish to avoid the danger that redistributors of a free program will
individually obtain patent licenses, in effect making the program
proprietary.  To prevent this, we have made it clear that any patent must
be licensed for everyone's free use or not licensed at all.

The precise terms and conditions for copying, distribution and
modification follow.

\begin{center}
{\Large \sc Terms and Conditions For Copying, Distribution and
  Modification}
\end{center}


%\renewcommand{\theenumi}{\alpha{enumi}}
\begin{enumerate}

\addtocounter{enumi}{-1}

\item 

This License applies to any program or other work which contains a notice
placed by the copyright holder saying it may be distributed under the
terms of this General Public License.  The ``Program'', below, refers to
any such program or work, and a ``work based on the Program'' means either
the Program or any derivative work under copyright law: that is to say, a
work containing the Program or a portion of it, either verbatim or with
modifications and/or translated into another language.  (Hereinafter,
translation is included without limitation in the term ``modification''.)
Each licensee is addressed as ``you''.

Activities other than copying, distribution and modification are not
covered by this License; they are outside its scope.  The act of
running the Program is not restricted, and the output from the Program
is covered only if its contents constitute a work based on the
Program (independent of having been made by running the Program).
Whether that is true depends on what the Program does.

\item You may copy and distribute verbatim copies of the Program's source
  code as you receive it, in any medium, provided that you conspicuously
  and appropriately publish on each copy an appropriate copyright notice
  and disclaimer of warranty; keep intact all the notices that refer to
  this License and to the absence of any warranty; and give any other
  recipients of the Program a copy of this License along with the Program.

You may charge a fee for the physical act of transferring a copy, and you
may at your option offer warranty protection in exchange for a fee.

\item

You may modify your copy or copies of the Program or any portion
of it, thus forming a work based on the Program, and copy and
distribute such modifications or work under the terms of Section 1
above, provided that you also meet all of these conditions:

\begin{enumerate}

\item 

You must cause the modified files to carry prominent notices stating that
you changed the files and the date of any change.

\item

You must cause any work that you distribute or publish, that in
whole or in part contains or is derived from the Program or any
part thereof, to be licensed as a whole at no charge to all third
parties under the terms of this License.

\item
If the modified program normally reads commands interactively
when run, you must cause it, when started running for such
interactive use in the most ordinary way, to print or display an
announcement including an appropriate copyright notice and a
notice that there is no warranty (or else, saying that you provide
a warranty) and that users may redistribute the program under
these conditions, and telling the user how to view a copy of this
License.  (Exception: if the Program itself is interactive but
does not normally print such an announcement, your work based on
the Program is not required to print an announcement.)

\end{enumerate}


These requirements apply to the modified work as a whole.  If
identifiable sections of that work are not derived from the Program,
and can be reasonably considered independent and separate works in
themselves, then this License, and its terms, do not apply to those
sections when you distribute them as separate works.  But when you
distribute the same sections as part of a whole which is a work based
on the Program, the distribution of the whole must be on the terms of
this License, whose permissions for other licensees extend to the
entire whole, and thus to each and every part regardless of who wrote it.

Thus, it is not the intent of this section to claim rights or contest
your rights to work written entirely by you; rather, the intent is to
exercise the right to control the distribution of derivative or
collective works based on the Program.

In addition, mere aggregation of another work not based on the Program
with the Program (or with a work based on the Program) on a volume of
a storage or distribution medium does not bring the other work under
the scope of this License.

\item
You may copy and distribute the Program (or a work based on it,
under Section 2) in object code or executable form under the terms of
Sections 1 and 2 above provided that you also do one of the following:

\begin{enumerate}

\item

Accompany it with the complete corresponding machine-readable
source code, which must be distributed under the terms of Sections
1 and 2 above on a medium customarily used for software interchange; or,

\item

Accompany it with a written offer, valid for at least three
years, to give any third party, for a charge no more than your
cost of physically performing source distribution, a complete
machine-readable copy of the corresponding source code, to be
distributed under the terms of Sections 1 and 2 above on a medium
customarily used for software interchange; or,

\item

Accompany it with the information you received as to the offer
to distribute corresponding source code.  (This alternative is
allowed only for noncommercial distribution and only if you
received the program in object code or executable form with such
an offer, in accord with Subsection b above.)

\end{enumerate}


The source code for a work means the preferred form of the work for
making modifications to it.  For an executable work, complete source
code means all the source code for all modules it contains, plus any
associated interface definition files, plus the scripts used to
control compilation and installation of the executable.  However, as a
special exception, the source code distributed need not include
anything that is normally distributed (in either source or binary
form) with the major components (compiler, kernel, and so on) of the
operating system on which the executable runs, unless that component
itself accompanies the executable.

If distribution of executable or object code is made by offering
access to copy from a designated place, then offering equivalent
access to copy the source code from the same place counts as
distribution of the source code, even though third parties are not
compelled to copy the source along with the object code.

\item
You may not copy, modify, sublicense, or distribute the Program
except as expressly provided under this License.  Any attempt
otherwise to copy, modify, sublicense or distribute the Program is
void, and will automatically terminate your rights under this License.
However, parties who have received copies, or rights, from you under
this License will not have their licenses terminated so long as such
parties remain in full compliance.

\item
You are not required to accept this License, since you have not
signed it.  However, nothing else grants you permission to modify or
distribute the Program or its derivative works.  These actions are
prohibited by law if you do not accept this License.  Therefore, by
modifying or distributing the Program (or any work based on the
Program), you indicate your acceptance of this License to do so, and
all its terms and conditions for copying, distributing or modifying
the Program or works based on it.

\item
Each time you redistribute the Program (or any work based on the
Program), the recipient automatically receives a license from the
original licensor to copy, distribute or modify the Program subject to
these terms and conditions.  You may not impose any further
restrictions on the recipients' exercise of the rights granted herein.
You are not responsible for enforcing compliance by third parties to
this License.

\item
If, as a consequence of a court judgment or allegation of patent
infringement or for any other reason (not limited to patent issues),
conditions are imposed on you (whether by court order, agreement or
otherwise) that contradict the conditions of this License, they do not
excuse you from the conditions of this License.  If you cannot
distribute so as to satisfy simultaneously your obligations under this
License and any other pertinent obligations, then as a consequence you
may not distribute the Program at all.  For example, if a patent
license would not permit royalty-free redistribution of the Program by
all those who receive copies directly or indirectly through you, then
the only way you could satisfy both it and this License would be to
refrain entirely from distribution of the Program.

If any portion of this section is held invalid or unenforceable under
any particular circumstance, the balance of the section is intended to
apply and the section as a whole is intended to apply in other
circumstances.

It is not the purpose of this section to induce you to infringe any
patents or other property right claims or to contest validity of any
such claims; this section has the sole purpose of protecting the
integrity of the free software distribution system, which is
implemented by public license practices.  Many people have made
generous contributions to the wide range of software distributed
through that system in reliance on consistent application of that
system; it is up to the author/donor to decide if he or she is willing
to distribute software through any other system and a licensee cannot
impose that choice.

This section is intended to make thoroughly clear what is believed to
be a consequence of the rest of this License.

\item
If the distribution and/or use of the Program is restricted in
certain countries either by patents or by copyrighted interfaces, the
original copyright holder who places the Program under this License
may add an explicit geographical distribution limitation excluding
those countries, so that distribution is permitted only in or among
countries not thus excluded.  In such case, this License incorporates
the limitation as if written in the body of this License.

\item
The Free Software Foundation may publish revised and/or new versions
of the General Public License from time to time.  Such new versions will
be similar in spirit to the present version, but may differ in detail to
address new problems or concerns.

Each version is given a distinguishing version number.  If the Program
specifies a version number of this License which applies to it and ``any
later version'', you have the option of following the terms and conditions
either of that version or of any later version published by the Free
Software Foundation.  If the Program does not specify a version number of
this License, you may choose any version ever published by the Free Software
Foundation.

\item
If you wish to incorporate parts of the Program into other free
programs whose distribution conditions are different, write to the author
to ask for permission.  For software which is copyrighted by the Free
Software Foundation, write to the Free Software Foundation; we sometimes
make exceptions for this.  Our decision will be guided by the two goals
of preserving the free status of all derivatives of our free software and
of promoting the sharing and reuse of software generally.

\begin{center}
{\Large\sc
No Warranty
}
\end{center}

\item
{\sc Because the program is licensed free of charge, there is no warranty
for the program, to the extent permitted by applicable law.  Except when
otherwise stated in writing the copyright holders and/or other parties
provide the program ``as is'' without warranty of any kind, either expressed
or implied, including, but not limited to, the implied warranties of
merchantability and fitness for a particular purpose.  The entire risk as
to the quality and performance of the program is with you.  Should the
program prove defective, you assume the cost of all necessary servicing,
repair or correction.}

\item
{\sc In no event unless required by applicable law or agreed to in writing
will any copyright holder, or any other party who may modify and/or
redistribute the program as permitted above, be liable to you for damages,
including any general, special, incidental or consequential damages arising
out of the use or inability to use the program (including but not limited
to loss of data or data being rendered inaccurate or losses sustained by
you or third parties or a failure of the program to operate with any other
programs), even if such holder or other party has been advised of the
possibility of such damages.}

\end{enumerate}


\begin{center}
{\Large\sc End of Terms and Conditions}
\end{center}


\pagebreak[2]

\section*{Appendix: How to Apply These Terms to Your New Programs}

If you develop a new program, and you want it to be of the greatest
possible use to the public, the best way to achieve this is to make it
free software which everyone can redistribute and change under these
terms.

  To do so, attach the following notices to the program.  It is safest to
  attach them to the start of each source file to most effectively convey
  the exclusion of warranty; and each file should have at least the
  ``copyright'' line and a pointer to where the full notice is found.

\begin{quote}
one line to give the program's name and a brief idea of what it does. \\
Copyright (C) yyyy  name of author \\

This program is free software; you can redistribute it and/or modify
it under the terms of the GNU General Public License as published by
the Free Software Foundation; either version 2 of the License, or
(at your option) any later version.

This program is distributed in the hope that it will be useful,
but WITHOUT ANY WARRANTY; without even the implied warranty of
MERCHANTABILITY or FITNESS FOR A PARTICULAR PURPOSE.  See the
GNU General Public License for more details.

You should have received a copy of the GNU General Public License
along with this program; if not, write to the Free Software
Foundation, Inc., 51 Franklin Street, Fifth Floor, Boston, MA  02110-1301, USA.
\end{quote}

Also add information on how to contact you by electronic and paper mail.

If the program is interactive, make it output a short notice like this
when it starts in an interactive mode:

\begin{quote}
Gnomovision version 69, Copyright (C) yyyy  name of author \\
Gnomovision comes with ABSOLUTELY NO WARRANTY; for details type `show w'. \\
This is free software, and you are welcome to redistribute it
under certain conditions; type `show c' for details.
\end{quote}


The hypothetical commands {\tt show w} and {\tt show c} should show the
appropriate parts of the General Public License.  Of course, the commands
you use may be called something other than {\tt show w} and {\tt show c};
they could even be mouse-clicks or menu items---whatever suits your
program.

You should also get your employer (if you work as a programmer) or your
school, if any, to sign a ``copyright disclaimer'' for the program, if
necessary.  Here is a sample; alter the names:

\begin{quote}
Yoyodyne, Inc., hereby disclaims all copyright interest in the program \\
`Gnomovision' (which makes passes at compilers) written by James Hacker. \\

signature of Ty Coon, 1 April 1989 \\
Ty Coon, President of Vice
\end{quote}


This General Public License does not permit incorporating your program
into proprietary programs.  If your program is a subroutine library, you
may consider it more useful to permit linking proprietary applications
with the library.  If this is what you want to do, use the GNU Library
General Public License instead of this License.

\end{document}
